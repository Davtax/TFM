% Chapter 1

\chapter{Introduction} % Main chapter title

\label{sec:Introduction} % For referencing the chapter elsewhere, use \ref{Chapter1} 

%----------------------------------------------------------------------------------------

% Define some commands to keep the formatting separated from the content 
\newcommand{\keyword}[1]{\textbf{#1}}
\newcommand{\tabhead}[1]{\textbf{#1}}
\newcommand{\code}[1]{\texttt{#1}}
\newcommand{\file}[1]{\texttt{\bfseries#1}}
\newcommand{\option}[1]{\texttt{\itshape#1}}

%----------------------------------------------------------------------------------------



%Besides, the use of scalable solid states devices with circuits that host the qubits and their control electronic on the same chip increases dramatically the density of qubits per unit of area. However, there exist mechanisms like the spin-orbit or the hyperfine-interaction that may limit us in performing certain operations on our qubit. In the recent years it has been proved that this non-desired effects can be reduced in some devices like silicon quantum dots with electrons. Yet due to the presence of a indirect gap in it's band structure, the coupling of electrons in this material with photons can be highly difficult, this being a crucial point when transporting quantum information over long distances.
%
%In recent years there has been growing interest in considering valence band heavy holes (HH) spines as quantum bits. One of the main reasons is the drastic reduction of the hyperfine interaction due to the p-symmetry of the valence band. In addition, the spin-orbit interaction is much more intense than in electrons, and it could be an advantage rarely explored nowadays. This interaction implies a new degree of freedom when manipulating the system and performing operations with it.

In 1982, Richard Feynman and Yuri Manin proposed \cite{Feynman1982} the idea of using the fundamental principles of quantum mechanics to solve a wide range of problems which have a computational cost too high to be solved efficiently with the best supercomputers available. This was the beginning of quantum computing. Since then, countless advances have been made in this field, both on a theoretical as well as an experimental level. It was only a few months ago that John Martinis' group managed to achieve the much desired quantum supremacy \cite{Arute2019}. This milestone proposed years ago by John Preskill \cite{Preskill2012} has a somewhat unfortunate name, because it only required to solve one problem, regardless of its usefulness, more efficiently than a classical computer. Nonetheless, this is a big step forward that brings us closer to building a quantum computer capable of solving complex problems from mathematics\cite{Wang2005}, chemistry\cite{ArgueelloLuengo2019}, material science\cite{Itoh2014}, biology\cite{Arndt2009} and even finances\cite{Haven2002}.\\

The quantum computer differs from the classical digital computer in the sense that instead of using a binary digit (bit) to represent Boolean logic states of 0 or 1, it uses a qubit that can represent an arbitrary superposition of both states $\ket{\psi}=\alpha\ket{0} +\beta\ket{1}$. The qubits are extremely fragile and difficult to operate and read-out, since in order to preserve the coherence of the superposition all the operations must be non-destructive. This typically require cryogenic temperatures to have decoherence and relaxation times large enough to perform a substantial number of operation on the qubits. This, in combination with the difficulty of building devices with a high number of qubits is what defined the current era, known as the Noisy Intermediate-Scale Quantum (NISQ) era.\\

Many efforts have been made in order to find the best system in which encode the quantum information. Among all the proposed technologies one can highlight the following ones. The most promising approach form the point of view of decoherence time is based on trapped ions\cite{Debnath2016}, unfortunately the manipulations in difficult. Solid-state methods based on transmon superconducting charge qubits are currently the dominant devices. This is in fact the technology used by IBM in their quantum computers, which are available for anyone to use via internet\cite{McKay2018}. The main problem for this devices is the scalability, the largest chips have a total of 72 qubits, housed in a room-sized cryostat at 15 mK. This number is very far away from the prediction of billions of qubits needed for practical quantum information processing\cite{Fowler2012}. There is also a promising theoretical proposal based on topological quantum computation by braiding Majorana fermions\cite{Nickerson2013}, however, this has not been experimentally realized to date. One of the most promising systems are semiconductor quantum dots (QD), taking account spin, valley and orbital degrees of freedom. Most of these works rely on the manipulation of either spin or both spin and charge of the particle (hybrid qubits). One advantage of this solid-state devices is the ability to perform all-electrostatic operations, because state-of-the-art technologies have achieved very high precision control pulses. A universal set of quantum gates has been demonstrated in GaAs electronic double QDs (DQD), performing single and double qubits gates\cite{Petta2010,Ribeiro2009}. Despite all the advantages, hyperfine interactions limit the coherence times coupling the electron in the potential wells with the nuclei in the GaAs substrate. One practical solution is the use of isotope purification in silicon QD's, what reduced dramatically the hyperfine interaction with the electron's spines. In this device it has been proved the possibility to implement a universal set of quantum gates\cite{Masuda2018}. Yet due to the presence of a indirect gap in the band structure of the material, the coupling of electrons with photons can be highly difficult. This coupling is primordial in order to transfer the quantum information over long distances. In recent years has been growing interest in considering valence band heavy holes (HH) spines in GaAs. One of the main reasons is the drastic reduction of the hyperfine interaction due to the p-symmetry of the valence band. In addition, the spin-orbit coupling (SOC) is much more intense than with electrons, which adds a new degree of freedom rarely explored nowadays. This work is focused in the study of these particles, hosted in either a DQD or a linear triple QD (TQD) array in GaAs.\\

Once the system in which encode the qubits is chosen, the next step is to being able to perform different operations. A very useful technique to initialize and manipulate states are the quantum adiabatic processes. With this the populations keep constant in the instantaneous eigenvalues of a time-dependent Hamiltonian, but the typical operation times are too long. This is a source of error because decoherence and noise drive the system way from the desired final state. The pulses may be speed up, but the presence of avoided crossings constitute a new source of leakage for the adiabatic state. It is necessary a protocol that allows quick transferences but that take care of the possible excitation, a feasible family of solutions are the so-called shortcuts to adiabaticity (STA). In a really life implementation there exist different errors which deviate the pulses from their ideal trajectories, this is we the protocols that we will use must be robust against systematic errors and stochastic noise.\\

\begin{center}
	\textcolor{red}{CHAPTERS STRUCTURE}
\end{center}








